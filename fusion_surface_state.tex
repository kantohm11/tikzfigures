\documentclass[crop,tikz]{standalone}% 'crop' is the default for v1.0, before it was 'preview'
\usepackage[]{xcolor}
\definecolor{orangered}{RGB}{181, 23, 0}
\definecolor{greenblue}{RGB}{0, 108, 101}
\usetikzlibrary{arrows,positioning,shapes}
\usetikzlibrary{decorations.pathreplacing,decorations.markings}
\usetikzlibrary{calc}
\tikzset{
  % style to apply some styles to each segment of a path
  on each segment/.style={
    decorate,
    decoration={
      show path construction,
      moveto code={},
      lineto code={
        \path [#1]
        (\tikzinputsegmentfirst) -- (\tikzinputsegmentlast);
      },
      curveto code={
        \path [#1] (\tikzinputsegmentfirst)
        .. controls
        (\tikzinputsegmentsupporta) and (\tikzinputsegmentsupportb)
        ..
        (\tikzinputsegmentlast);
      },
      closepath code={
        \path [#1]
        (\tikzinputsegmentfirst) -- (\tikzinputsegmentlast);
      },
    },
  },
  % style to add an arrow in the middle of a path
  mid arrow/.style={postaction={decorate,decoration={
        markings,
        mark=at position .5 with {\arrow[#1]{stealth}}
      }}},
}

% adopted and modified from https://tex.stackexchange.com/questions/107409/how-to-draw-a-3d-hexagonal-structure-with-tikz, the one by alain-matthes
\newcommand\hexgridv[2][]{%
\begin{scope}[%
  #1,
  every node/.style={anchor=west,regular polygon, regular polygon sides=6,draw=none,inner sep=0.5cm},transform shape
]
\node (A#2) {};
\node (B#2) at ([xshift=-\pgflinewidth,yshift=-\pgflinewidth]A#2.corner 1) {};
\node (C#2) at ([xshift=-\pgflinewidth]B#2.corner 5) {};
\node (D#2) at ([xshift=-\pgflinewidth]A#2.corner 5) {};
\node (E#2) at ([xshift=-\pgflinewidth]D#2.corner 5) {};
%\foreach \hex in {A,...,E}
%{
%  \foreach \corn in {1,...,6}
%    \draw[fill=white] (\hex#2.corner \corn) circle (2pt); 
%}
\end{scope}
}

\newcommand\pgfmathsinandcos[3]{% 
  \pgfmathsetmacro#1{sin(#3)}% 
  \pgfmathsetmacro#2{cos(#3)}% 
}

\begin{document}


\newcommand{\sethexcoord}[2]{
\pgfmathsetmacro{\hexcoordx}{(#1-#2)*sqrt(3)/2}
\pgfmathsetmacro{\hexcoordy}{#2/2*3-mod(#1,2)/2}
}

\newcommand{\sethexpoints}[1]{
%\yy=0
\foreach \xx in {0,1, ..., 4}{
\sethexcoord{\xx}{0}
\coordinate (h\xx0#1) at (\hexcoordx,\hexcoordy);
}
%\yy=1
\foreach \xx in {0,1, ..., 6}{
\sethexcoord{\xx}{1}
\coordinate (h\xx1#1) at (\hexcoordx,\hexcoordy);
}
%\yy=2
\foreach \xx in {1,2, ..., 7}{
\sethexcoord{\xx}{2}
\coordinate (h\xx2#1) at (\hexcoordx,\hexcoordy);
}
}

\pgfmathsetmacro\angFuite{65}
\pgfmathsetmacro\coeffReduc{.5}
\pgfmathsinandcos\sint\cost{\angFuite}
%\pgfmathparse{#2/2*3-mod(#1,2)/2}\pgfmathresult)}


%draw hex
\newcommand{\sethexpath}[3]{
  \pgfmathtruncatemacro{\xx}{#1}
  \pgfmathtruncatemacro{\xxp}{\xx+1}
  \pgfmathtruncatemacro{\xxpp}{\xx+2}
  \pgfmathtruncatemacro{\xxppp}{\xx+3}
  \pgfmathtruncatemacro{\yy}{#2}
  \pgfmathtruncatemacro{\yyp}{\yy+1}
  \pgfmathtruncatemacro{\zz}{#3}
}
\newcommand{\drawhex}[1]{
 \draw[#1](h\xx\yy\zz)--(h\xxp\yy\zz)--(h\xxpp\yy\zz)--(h\xxppp\yyp\zz)--(h\xxpp\yyp\zz)--(h\xxp\yyp\zz)--cycle;
}

%draw rectangle streatching along z direction 
\newcommand{\setrectanglepath}[6]{
  \pgfmathtruncatemacro{\xx}{#1}
  \pgfmathtruncatemacro{\yy}{#2}
  \pgfmathtruncatemacro{\zz}{#3}
  \pgfmathtruncatemacro{\xxp}{#4}
  \pgfmathtruncatemacro{\yyp}{#5}
  \pgfmathtruncatemacro{\zzp}{#6}
}
\newcommand{\drawrectangle}[1]{
 \draw[#1](h\xx\yy\zz)--(h\xxp\yyp\zz)--(h\xxp\yyp\zzp)--(h\xx\yy\zzp)--cycle;
}

\begin{tikzpicture}[scale=1.5,
current plane/.estyle={cm={1,0,\coeffReduc*\cost,\coeffReduc*\sint,(0,#1)}},
]

%set hex points

\begin{scope}[current plane=0.5cm,]
\sethexpoints{0}
\end{scope}
\begin{scope}[current plane=.5cm,]
\sethexpoints{1}
\end{scope}
\begin{scope}[current plane=.5cm,]
\sethexpoints{2}
\end{scope}
\begin{scope}[current plane=.5cm,]
\sethexpoints{3}
\end{scope}
\begin{scope}[current plane=2cm,]
\sethexpoints{4}
\end{scope}

\begin{scope}[
    vertfill/.style={fill=gray,opacity=0.2,draw=none},
    vertdrawf/.style={orangered,mid arrow,thick},
    vertdrawg/.style={greenblue,mid arrow,thick},
    fillhex/.style={fill=#1, fill opacity=0.3,draw=#1!70!white},
    ]
\foreach \xx in {1,2,...,6}{
  \pgfmathtruncatemacro{\xxp}{\xx+1}
  \draw[vertfill] (h\xx24)--(h\xx20)-- (h\xxp20)-- (h\xxp24) -- cycle;
}
\foreach \xx in {1,3,...,7}{
  \draw[vertdrawf] (h\xx24)--(h\xx20);
}
\foreach \xx in {2,4,...,6}{
  \draw[vertdrawg] (h\xx20)--(h\xx24);
}
\setrectanglepath{0}{1}{4}{1}{2}{0}
\drawrectangle{vertfill}
\sethexpath{0}{1}{4}
\drawhex{fillhex=blue}
\setrectanglepath{2}{1}{4}{3}{2}{0}
\drawrectangle{vertfill}
\sethexpath{2}{1}{4}
\drawhex{fillhex=blue}
\setrectanglepath{4}{1}{4}{5}{2}{0}
\drawrectangle{vertfill}
\sethexpath{4}{1}{4}
\drawhex{fillhex=blue}
\setrectanglepath{6}{1}{4}{7}{2}{0}
\drawrectangle{vertfill}

\foreach \xx in {0,1,...,5}{
  \pgfmathtruncatemacro{\xxp}{\xx+1}
  \draw[vertfill] (h\xx14)--(h\xx10)-- (h\xxp10)-- (h\xxp14) -- cycle;
}
\foreach \xx in {1,3,...,5}{
  \draw[vertdrawf] (h\xx14)--(h\xx10);
}
\foreach \xx in {0,2,...,6}{
  \draw[vertdrawg] (h\xx10)--(h\xx14);
}
\setrectanglepath{0}{0}{4}{1}{1}{0}
\drawrectangle{vertfill}
\sethexpath{0}{0}{4}
\drawhex{fillhex=blue}
\setrectanglepath{2}{0}{4}{3}{1}{0}
\drawrectangle{vertfill}
\sethexpath{2}{0}{4}
\drawhex{fillhex=blue}
\setrectanglepath{4}{0}{4}{5}{1}{0}
\drawrectangle{vertfill}


\foreach \xx in {0,1,...,3}{
  \pgfmathtruncatemacro{\xxp}{\xx+1}
  \draw[vertfill] (h\xx04)--(h\xx00)-- (h\xxp00)-- (h\xxp04) -- cycle;
}
\foreach \xx in {1,3}{
  \draw[vertdrawf] (h\xx04)--(h\xx00);
}
\foreach \xx in {0,2,4}{
  \draw[vertdrawg] (h\xx00)--(h\xx04);
}
\end{scope}
  
\node[anchor=north] at (h100) {$\textcolor{orangered}{f}$};
\node[anchor=north] at (h000) {$\textcolor{greenblue}{g}$};

%add margins to display the arrows at the edges
\node at ([xshift=-.1] h010) {};
\node at ([xshift=.1] h720) {};

%Gamma labels
\node[xslant=\cost] at ($(h124)!.5!(h214)$) {$\Gamma_i$};
\node[xslant=\cost] at ($(h324)!.5!(h414)$) {$\Gamma_j$};
\node[xslant=\cost] at ($(h524)!.5!(h614)$) {$\Gamma_k$};
\node[xslant=\cost] at ($(h114)!.5!(h204)$) {$\Gamma_\ell$};
\node[xslant=\cost] at ($(h314)!.5!(h404)$) {$\Gamma_m$};

%Gamma edge labels
\node[xslant=\cost] at ($(h324)!.5!(h214)$) {$\Gamma_{ij}$};
\node[xslant=\cost] at ($(h524)!.5!(h414)$) {$\Gamma_{jk}$};
\node[xslant=\cost] at ($(h114)!.5!(h214)$) {$\Gamma_{i\ell}$};
\node[xslant=\cost] at ($(h214)!.5!(h314)$) {$\Gamma_{j\ell}$};
\node[xslant=\cost] at ($(h314)!.5!(h414)$) {$\Gamma_{jm}$};
\node[xslant=\cost] at ($(h414)!.5!(h514)$) {$\Gamma_{km}$};
\node[xslant=\cost] at ($(h314)!.5!(h204)$) {$\Gamma_{\ell m}$};

\node[] at ($(h004)!.5!(h100)$) {$\lambda$};
\node[] at ($(h104)!.5!(h200)$) {$\rho$};
\node[yslant=\sint] at ($(h204)!.5!(h310)$) {$\sigma$};
  
\end{tikzpicture}
\end{document}
